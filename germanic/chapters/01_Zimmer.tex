\documentclass[output=paper]{langsci/langscibook} 
\author{Christian Zimmer\affiliation{Freie Universität Berlin} and Horst J. Simon\affiliation{Freie Universität Berlin}}
\title{Contact settings involving Germanic languages}
\abstract{In this chapter, we outline the scope and the main aims of this volume. First, we briefly sketch the diversity of contact settings involving German(ic) varieties and the according research history. This sets the scene for a brief overview of the contributions included.}

\IfFileExists{../localcommands.tex}{
  % add all extra packages you need to load to this file

\usepackage{tabularx,multicol}
\usepackage{url}
\urlstyle{same}

\usepackage{listings}
\lstset{basicstyle=\ttfamily,tabsize=2,breaklines=true}

\usepackage{langsci-optional}
\usepackage{langsci-lgr}
\usepackage{langsci-gb4e}
% \usepackage{langsci-plots}
\usepackage{pgfplots}

\usepackage{siunitx}
\sisetup{group-digits=false}

\usepackage{amssymb}% http://ctan.org/pkg/amssymb
\usepackage{pifont}% http://ctan.org/pkg/pifont
\newcommand{\cmark}{\ding{51}}%
\newcommand{\xmark}{\ding{55}}%
% \usepackage[disable]{todonotes}
\usepackage{todonotes}

  \newcommand*{\orcid}{}
\newcommand{\hoederN}{n̡} 
  %% hyphenation points for line breaks
%% Normally, automatic hyphenation in LaTeX is very good
%% If a word is mis-hyphenated, add it to this file
%%
%% add information to TeX file before \begin{document} with:
%% %% hyphenation points for line breaks
%% Normally, automatic hyphenation in LaTeX is very good
%% If a word is mis-hyphenated, add it to this file
%%
%% add information to TeX file before \begin{document} with:
%% %% hyphenation points for line breaks
%% Normally, automatic hyphenation in LaTeX is very good
%% If a word is mis-hyphenated, add it to this file
%%
%% add information to TeX file before \begin{document} with:
%% \include{localhyphenation}
\hyphenation{
affri-ca-te
affri-ca-tes 
}
\hyphenation{
affri-ca-te
affri-ca-tes 
}
\hyphenation{
affri-ca-te
affri-ca-tes 
} 
  \togglepaper[1]%%chapternumber
}{}



\begin{document}
\maketitle

\section{Germanic varieties in language contact: Scenarios and research traditions} %1 
\label{sec:zimmer:1}

It is well-known that contact between speakers of different languages or varieties leads to dynamics in many respects.\footnote{This work was funded by the \textit{Deutsche} \textit{Forschungsgemeinschaft} (DFG, German Research Foundation) – WI 2155/9-1; SI 750/4-1/2; ZI~1793/1\nobreakdash-2. Furthermore, we thank about two dozens of colleagues, who were generous with their time when they peer-reviewed papers for the present volume.} From a grammatical perspective, especially contact between closely related languages/varieties fosters contact-induced innovations (as put forward by, e.g., \citealt{thomason_contact-induced_2014}). The evaluation of such innovations reveals speakers’ attitudes and is in turn an important aspect of the sociolinguistic dynamics linked to language contact. 

In this volume, we assemble studies on such settings where typologically congruent languages are in contact, i.e. language contact within the Germanic branch of the Indo-European language family. Languages involved include Afrikaans, Danish, English, Frisian, (Low and High) German, and Yiddish. The main focus is on constellations where a variety of German is involved.\footnote{This is why we use the term ‘German(ic)’ in this book: We do not want to constrain ourselves to scenarios involving a German variety, but at the same time we do not want to pretend that we assemble studies on Germanic contact varieties in a balanced way. There are no further implications of this term.}  These scenarios are multifaceted. Apart from some basic commonalities (such as the language(s) involved) these constellations differ in many respects. For example, there are settings where language contact results from emigration from Europe, e.g. to Africa (see, e.g., \citealt{wiese_german_2017}), to the Americas (see, e.g., \citealt{johannessen_germanic_2015}), to Australia (see, e.g., \citealt{riehl_language_2015}), or to Melanesia (see., e.g. \citealt{maitz_unserdeutsch_nodate}). These settings can again be differentiated with regard to the extent and the role that colonialism played in the migration process. For example, the German-speaking minority in Namibia has its roots in the purposeful colonisation of South-Western Africa by the German government, resulting in the colony \textit{Deutsch-Südwestafrika}. In contrast, other migration movements (such as the ones to North America) cannot be described as the result of concrete colonialist efforts (in the narrow sense), but are part of the more general colonial expansion of Europeans. Apart from that, language contact, of course, also results from immigration to Europe (cf., e.g., \citealt{wiese_what_2013}). In addition, in many cases no (recent) migration is involved; here two or more varieties are often in long-term contact (cf., e.g. \citetv{chapters/02_Hoeder}) %(cf., e.g., Höder, this volume).

So far, studies on language contact with German have often been separated according to the different migration scenarios at hand, which resulted in somewhat different research traditions. For example, the so-called \textit{Sprachinselforschung} (research on ‘language islands’) has mainly been concerned with settings caused by emigration from the continuous German-speaking area in Central Europe to locations in Central and Eastern Europe and overseas, thus resulting in some variety of German abroad. However, from a linguistic point of view it does not seem to be necessary to distinguish categorically between contact scenarios within and outside of Central Europe if one thoroughly considers the impact of sociolinguistic circumstances, including the ecology of the languages involved (such as, for instance, German being the majority language and the monolingual habitus in Germany, but completely different constellations elsewhere; see \citealt{haugen_ecology_1972} for the concept of language ecology). 

%In this volume we focus on language contact as such, not on specific migration scenarios. Hence, we assemble studies on language contact within and outside of Germany. For instance, \citetv{chapters/08_Rocker}, studies heritage language use in the United States, whilst Höder (this volume) and Gregersen \& Langer (this volume) focus on language contact in Northern Germany (and Denmark). Recent studies have revealed striking similarities between different varieties of German irrespective of their differing socio-historic backgrounds (and the respective contact languages), see, e.g., \citet{wiese_deutsch_2014}.\footnote{See also \citet{rosenberg_comparative_2003} for some revealing insights from “comparative speech island research”.} This supports the idea that the crucial aspect is language contact as such and that grammatical and sociolinguistic dynamics are comparable across contact scenarios in different parts of the world.

In this volume we focus on language contact as such, not on specific migration scenarios. Hence, we assemble studies on language contact within and outside of Germany. For instance, \citetv{chapters/08_Rocker}, studies heritage language use in the United States, whilst \citetv{chapters/02_Hoeder} and \citetv{chapters/07_Gregersen} focus on language contact in Northern Germany (and Denmark). Recent studies have revealed striking similarities between different varieties of German irrespective of their differing socio-historic backgrounds (and the respective contact languages), see, e.g., \citet{wiese_deutsch_2014}.\footnote{See also \citet{rosenberg_comparative_2003} for some revealing insights from “comparative speech island research”.} This supports the idea that the crucial aspect is language contact as such and that grammatical and sociolinguistic dynamics are comparable across contact scenarios in different parts of the world.

%German(ic) contact varieties do not only differ in their geographical location and their socio-historical background but also with regard to their vitality. On the one hand, there are instances of a completed language shift. For example, Low German was given up in North America (see, e.g., \citetv{chapters/08_Rocker}) and there are many other communities in that part of the world, where a language shift from different Germanic languages to English is imminent (cf., e.g., \citealt{page_moribund_2015}). On the other hand, there are also examples for persistent language maintenance in North America (cf., e.g., \citealt{Louden2016} on Pennsylvania Dutch) and elsewhere (cf., e.g., \citealt{shah_german_2018} and \citealt{rosenberg_lateinamerika_2018} on German in Namibia and in Latin America, respectively). This is often (but not always) linked to religious affiliations, which support separation from other surrounding groups. And finally, there are of course many intermediate cases (cf., e.g., Gregersen \& Langer (this volume) on efforts in Frisia to prevent language shift). The vitality of German(ic) varieties as spoken by minorities is closely linked to the institutional support from which these varieties benefit. This has a strong impact on where and when a language is used. Questions $-$ which are highly relevant with regard to language maintenance and shift $-$ include: Is the minority language used in private homes only? Is there a written form of the language in use? Are there (still) newspapers, radio and TV programmes, religious services, schools, and/or social media users who make use of the minority language? A reduction of domains can precede language shift, but this does not necessarily have to be the case. Also in this respect, the varieties at hand differ significantly. For example, German-language newspapers in North America were typically discontinued, or they switched to English during the 20\textsuperscript{th} century (cf. Rocker (this volume)), whilst the Namibian German-language newspaper \textit{Allgemeine} \textit{Zeitung} is still in daily print (cf., e.g., \citealt{shah_german_2018}).

German(ic) contact varieties do not only differ in their geographical location and their socio-historical background but also with regard to their vitality. On the one hand, there are instances of a completed language shift. For example, Low German was given up in North America (see, e.g., \citetv{chapters/08_Rocker}) and there are many other communities in that part of the world, where a language shift from different Germanic languages to English is imminent (cf., e.g., \citealt{page_moribund_2015}). On the other hand, there are also examples for persistent language maintenance in North America (cf., e.g., \citealt{Louden2016} on Pennsylvania Dutch) and elsewhere (cf., e.g., \citealt{shah_german_2018} and \citealt{rosenberg_lateinamerika_2018} on German in Namibia and in Latin America, respectively). This is often (but not always) linked to religious affiliations, which support separation from other surrounding groups. And finally, there are of course many intermediate cases (cf., e.g., \citetv{chapters/07_Gregersen} on efforts in Frisia to prevent language shift). The vitality of German(ic) varieties as spoken by minorities is closely linked to the institutional support from which these varieties benefit. This has a strong impact on where and when a language is used. Questions $-$ which are highly relevant with regard to language maintenance and shift $-$ include: Is the minority language used in private homes only? Is there a written form of the language in use? Are there (still) newspapers, radio and TV programmes, religious services, schools, and/or social media users who make use of the minority language? A reduction of domains can precede language shift, but this does not necessarily have to be the case. Also in this respect, the varieties at hand differ significantly. For example, German-language newspapers in North America were typically discontinued, or they switched to English during the 20\textsuperscript{th} century (cf. \citetv{chapters/08_Rocker}), whilst the Namibian German-language newspaper \textit{Allgemeine} \textit{Zeitung} is still in daily print (cf., e.g., \citealt{shah_german_2018}).

%Another important aspect is of course the composition of the languages and varieties interacting with each other. Germanic languages are in contact with closely related languages (i.e. other Germanic varieties, e.g., Yiddish in contact with American English in the United States; see, e.g., Nove (this volume)), related languages (i.e. other Indo-European languages, such as German in contact with Brazilian Portuguese in Brazil, see, e.g., \citealt{rosenberg_comparative_2003}) and unrelated languages (e.g. German in contact with Hungarian in Hungary, cf. \citealt{knipf-komlosi_ungarn_2008}). Although we focus on the first type of setting in this volume, there is still a great variety of constellations to be examined. For example, these constellations differ in the number of languages involved. Many scenarios involve more than two major contact languages/varieties. This holds true especially (but not only) if we also consider non-standard varieties.\footnote{See, e.g., \citet{schirmunski_sprachgeschichte_1930}, \citet{trudgill_dialects_1986}, and \citet{rosenberg_dialect_2005} for studies on the dynamics induced by dialect contact.} In the Danish-German contact zone, for example, Standard Danish, Jutlandic Standard Danish, South Jutlandic, Standard German, North High German, and Low German interact (among other varieties, cf. Höder (this volume)).\footnote{Assuming that such varieties can be neatly distinguished.} See also the settings in Namibia and South Africa where (among others) German, Afrikaans, and English (including several non-standard varieties) are in close contact. In such cases we are dealing with contact of several closely related varieties (see, e.g., \citealt{zimmer_deutsch_2019}).

Another important aspect is of course the composition of the languages and varieties interacting with each other. Germanic languages are in contact with closely related languages (i.e. other Germanic varieties, e.g., Yiddish in contact with American English in the United States; see, e.g., \citetv{chapters/03_Nove}), related languages (i.e. other Indo-European languages, such as German in contact with Brazilian Portuguese in Brazil, see, e.g., \citealt{rosenberg_comparative_2003}) and unrelated languages (e.g. German in contact with Hungarian in Hungary, cf. \citealt{knipf-komlosi_ungarn_2008}). Although we focus on the first type of setting in this volume, there is still a great variety of constellations to be examined. For example, these constellations differ in the number of languages involved. Many scenarios involve more than two major contact languages/varieties. This holds true especially (but not only) if we also consider non-standard varieties.\footnote{See, e.g., \citet{schirmunski_sprachgeschichte_1930}, \citet{trudgill_dialects_1986}, and \citet{rosenberg_dialect_2005} for studies on the dynamics induced by dialect contact.} In the Danish-German contact zone, for example, Standard Danish, Jutlandic Standard Danish, South Jutlandic, Standard German, North High German, and Low German interact (among other varieties, cf. Höder \citetv{chapters/02_Hoeder}).\footnote{Assuming that such varieties can be neatly distinguished.} See also the settings in Namibia and South Africa where (among others) German, Afrikaans, and English (including several non-standard varieties) are in close contact. In such cases we are dealing with contact of several closely related varieties (see, e.g., \citealt{zimmer_deutsch_2019}).

The diversity of the different scenarios outlined above allows us to study many different aspects of the dynamics induced by language contact. With this volume, we hope to exploit this potential in order to shed some new light on the interplay of language contact, variation, change, and the concomitant sociolinguistic dynamics. Particularly, we hope to contribute to a better understanding of closely related varieties in contact. 

By doing so, we also aim to deepen research on German(ic) in language contact from a decidedly contact linguistic perspective. $-$ There is a long-standing tradition of research on Germanic in different contact settings. As mentioned above, the German \textit{Sprachinseln} (‘language islands’) in particular have been the focus of attention for a long time, beginning already in the 19\textsuperscript{th} century (see \citealt{Rosenberg2005} for an overview). However, research on these varieties has mostly been carried out in the context of descriptive dialectology, more specifically as \textit{Sprachinselforschung}, with a goal to investigate the preservation of inherited features. There was no genuine interest in language contact:

\begin{quote}
In German dialectology, language islands were predominantly investigated as relics of the past for the purpose of studies in language change. Most of the linguistic communities examined were rather small with restricted external communication. Since these conservative communities frequently preserved archaic features of German, they were seen as offering access to linguistic elements which had died out in the main German language area. […] The interest in language islands was built on a myth of purity and homogeneity. Language variation and language contact were considered more a source of data corruption than as a subject of research. (\citealt{Rosenberg2005}: 222–223)
\end{quote}

     Subsequently, the interest in language contact phenomena increased in the field. However, the original \textit{Sprachinsel} approach continued to have an effect. It is only recently that a re-orientation of the field can be observed, which was (at least partly) initiated through the programmatic article by \citet{mattheier_theorie_1994}. By now, discussions have broadened in scope by taking into account the concepts and methods that have been developed in the international literature on language contact and language variation (see, for example, \citealt{putnam_studies_2011}, \citealt{page_moribund_2015} and \citealt{boas_constructions_2018}). It is our aim to further exactly this line of research. In this volume, we assemble studies that…\\
\begin{itemize}
\item 
view language contact from a grammar-theoretical perspective (see the contribution by Steffen Höder),

\item 
focus on lesser studied contact settings (e.g. German in Namibia; see the contributions by Yannic Bracke, Henning Radke, and Britta Stuhl) 

\item 
make use of new corpus linguistic resources (see the contributions by Yannic Bracke and Britta Stuhl) or newly acquired data (see the contribution by Maike H. Rocker)

\item 
analyse data quantitatively (see, e.g., the contribution by Chaya R. Nove)

\item 
study language contact phenomena in computer-mediated communication (see the contributions by Johanna Gregersen \& Nils Langer and Henning Radke)

\item 
focus on the interplay of language use and language attitudes or ideologies (see, e.g., the contributions by Yannic Bracke, and Johanna Gregersen \& Nils Langer)
\end{itemize}

In the following section, we briefly outline the structure and the contributions of this volume. 

\section{The papers in this volume} %2 /
\label{sec:zimmer:2}

The volume at hand is mainly based on a selection of papers that were originally presented at the workshop \textit{German(ic)} \textit{in} \textit{language} \textit{contact:} \textit{Grammatical} \textit{and} \textit{sociolinguistic} \textit{dynamics}, which was held at Freie Universität Berlin (3–5 July 2019).\footnote{This workshop was organised by the members of the DFG-funded research project “Namdeutsch: The dynamics of German in the multilingual context of Namibia” (PIs: Horst Simon, Freie Universität Berlin, and Heike Wiese, Humboldt-Universität zu Berlin).}  The topics covered range from phonetics, morphology, and syntax to the use and perception of transferred lexical and grammatical material and issues related to language shift and maintenance. The volume brings together authors who share a general interest in language contact phenomena but work in different frameworks, such as scholars who are concerned with corpus linguistics, sociolinguistics, theoretical approaches to multilingualism etc. 

The book consists of two major sections. The first section focuses on grammatical aspects of language contact (including phonetics), whilst the contributions in the second section are mainly concerned with sociolinguistic dynamics. The first section starts with a contribution by {Steffen} {Höder,} who examines morphosyntactic arealisms in the Danish-German contact zone, i.e. features shared by a number of German and Danish varieties which have been shaped by consistent language contact. These features are addressed within the framework of Diasystematic Construction Grammar (DCxG). A core assumption of this approach is the idea that language-specificity is part of a construction’s pragmatic meaning and that constructicons comprise both language-specific and language-\textit{un}specific constructions (i.e. \textit{idioconstructions} and \textit{diaconstructions}). Höder claims that the proportion of diaconstructions in a multilingual constructicon increases constantly. The pertinent mechanisms are demonstrated with the help of selected arealisms, such as the \textsc{shall} future. 

The following two contributions are both concerned with phonetic phenomena in contact settings. The paper by {Chaya} {R.} {Nove} focuses on phonetic change within the community of Hasidic Yiddish speakers in New York City using the apparent time approach. To this extent, the phonetic systems of three different generations of Hasidic Yiddish-English bilinguals are compared (more specifically, the vowels /i, ɪ, u, ʊ, a/). It is shown that convergence can be observed in the younger generations to a greater extent. This main result is interpreted with reference to models of second language acquisition, with a special focus on the impact of the linguistic input.

The contribution by {Britta} {Stuhl} is the first of three papers in this volume studying the contact setting of German in Namibia. This setting does not only involve contact of closely related and (to a much lesser extent) unrelated languages (such as Afrikaans, English, German, Khoekhoegowab, and Oshiwambo), but also of dialects of German. Britta Stuhl focuses on the latter aspect. Her contribution centres around the question as to what extent features of Northern German varieties (which were used by a significant proportion of the German-speaking immigrants) have survived dialect levelling. Her corpus study reveals that Namibian German does indeed contain specifically Northern German phonetic features; the fact that some of these features are more frequently used by older speakers hints at an ongoing change. 

{Yannic} {Bracke} also examines language use within the Namibian German community, but he focuses on sociolinguistic aspects. He is concerned with the question of how the gender of speakers correlates with the use of transferred lexical items. The underlying assumption is that the use of loan words and the like (which are usually considered to be characteristic of non-standard language use) could be connected to a male stereotype. This idea is based on statements by community members. However, his corpus study (which comprises the elaboration of a sophisticated annotation system for transferred lexical items) shows that there is no consistent correlation of gender and language use in this respect. 

{Henning} {Radke} studies the use of informal Namibian German (i.e. \textit{Namdeutsch}) in computer-mediated communication. Most of the speakers he studies were born and raised in Namibia but live in Germany now. Within this diasporic group, Namdeutsch serves as an in-group marker. Transferred lexical material (from Afrikaans and English) plays a crucial role here. Radke compares the language use in two types of online communities quantitatively and qualitatively: single mode groups, which communicate only online, and mixed mode groups, which additionally meet face-to-face. Based on this comparison, he examines the interplay of communication mode, (multilingual) language use, and group cohesion.

While transferred lexical material has generally positive connotations within the Namibian German diaspora, the group examined by {Johanna} {Gregersen} {\&}  {Nils} {Langer} partially rejects such outcomes of language contact. In their study, Gregersen \& Langer focus on the assessment of borrowings by academic linguists working on North Frisian. Using examples from different types of scholarly and public discourse, they show that some of these scholars do not only describe but also evaluate language use. These evaluations can be seen in the context of linguistic purism: external influences on North Frisian are evaluated as a threat to the language. Such assessments are rather unusual in the context of academic linguistics. Gregersen \& Langer consider this to be specific to discourses on smaller languages.

The paper by {Maike} {H.} {Rocker} is the last contribution in this volume. It deals with heritage language use in print media, more specifically with Low German and High German correspondence letters to the \textit{Ostfriesen} \textit{Zeitung}, an East Frisian-American newspaper, which was published in the United States until 1971. She answers the following classic question: \textit{Who} \textit{writes} \textit{what} \textit{to} \textit{whom} \textit{in} \textit{which} \textit{language}? The results provide insights into a number of sociolinguistic aspects, such as the regional distribution of East Frisian communities in the United States, domains of Low German and High German language use, and the interrelation of pragmatic purpose and language choice. Finally, Rocker shows how the newspaper fostered a sense of East Frisian-American identity, which in turn facilitated language maintenance of both Low German and High German well into the 20\textsuperscript{th} century.

In sum, the papers collected in this volume reflect a wide array of current work in the thriving and fast-developing field of language contact studies with regard to German(ic). It is to be hoped that they give an idea of the range of insights that can be gained by applying methods and theories of contemporary language contact studies to a traditional sub-field of German(ic) linguistics.



{\sloppy\printbibliography[heading=subbibliography,notkeyword=this]}
\end{document}
